Der Mittelpunkt von Smarten Technologien und Sozialen Medien ist die Interaktion von deren Nutzern. Diese Technologien und Medien erleben seit Jahren ein rasantes Wachstum. In seinen Anf�ngen brauchte das Internet nur vier Jahre um 50 Millionen Nutzer zu erreichen. Facebook wuchs um 50 Millionen Nutzer in nur einem halben Jahr. \cite{Nair2011} Nutzerinteraktion bezieht sich dabei einerseits auf den Konsum von Inhalten, aber auch das Erstellen von Inhalten selbst. \cite{Alt2012}

Durch die Abwanderung zu neuen digitalen Medien verlieren die urspr�nglichen Printmedien, Radio oder Fernsehen zunehmend an Bedeutung. Echtzeit-Information wird ein integrales Element f�r Kundenverhalten. \cite{Hennig2010}

Durch diese fortschreitende Verbreitung von mobilen Technologien, wird Kommunikation auch zusehends Orts- und Situationsabh�ngig. \cite{Alt2012} Diese neuen M�glichkeiten der Interaktion ver�ndert auch das verhalten von Kunden. Dies gilt vor allem f�r die Art und Weise wie Information �ber Unternehmen und Produkte gesammelt und ausgetauscht werden. \cite{Hennig2010} Smarte Technologien und Soziale Applikationen geben Konsumenten die verschiedensten M�glichkeiten aktiv Informationen �ber Produkte und Dienste anzubieten. \cite{Foster2010} Durch Plattformen wie Ebay oder Amazon werden Konsumenten auch zu Verk�ufern ihrer eigenen Produkte.

Unternehmen m�ssen sich durch dieses ver�nderte Kommunikationsverhalten auf neue  Herausforderung im Bereich des Customer-Relationship-Managements (CRM) einstellen. Der mittelbare Kundenkontakt ver�ndert sich. Auߟendienstmitarbeiter, Kundenberater oder Call-Center-Agenten verlieren an Bedeutung. Kunden erwarten einen unmittelbaren mit den Unternehmen. \cite{Alt2012} Sie werden zu aktiven und stark vernetzten Partnern, welche auch die Rolle von Anbietern und Produzenten einnehmen k�nnen. Dies macht es f�r Unternehmen schwieriger das eigene Marken-Image und das Klima von Kundenbeziehungen zu kontrollieren \cite{Hennig2010}

Es ergeben sich viele neue Risiken f�r Unternehmen im Umgang mit sozialen Medien. Kunden geben den Erfahrungen und Meinungen anderer Konsumente h�heren Stellenwert und Glaubw�rdigkeit gegen�ber der Unternehmenskommunikation. Eskalierte Diskussionen in sozialen Medien k�nnen starke negativen Auswirkungen auf Unternehmen haben diese sp�t oder gar nicht daran teilnehmen. \cite{Alt2012}

Gleichzeitig gibt es auch viel Potential f�r neue Kommunikationsformen zwischen Kunden und Unternehmen. Unternehmen bekommen die M�glichkeit direkt mit ihren Kunden zu interagieren um sich �ber Kampagnen oder Probleme auszutauschen. Oftmals wird eine solche direkte Interaktion von den Kunden auch erwartet. \cite{Alt2012}

Besonders f�r Unternehmen mit Endkundenkontakt ist die Nutzung des SocialWeb eine wettbewerblichen Notwendigkeit. Gleichzeitig darf der Einsatz von neuen Medien, �ltere Konsumenten nicht ausschlie�en. Der Kontakt zu Konsumenten aus �lteren Bev�lkerungsschichten darf nicht vernachl�ssigt werden. \cite{Coughlin2007}